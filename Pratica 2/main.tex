\documentclass[]{beamer}
\usepackage{amsmath}
\usepackage{tikzpicture}

\title[Pratica 2]{Aula Pratica 2}
\author[P. Fagandini]{Paulo Fagandini}
\institute[ISCAL-IPL]{Lisbon Accounting and Business School}
\date{}

\begin{document}

\maketitle

\frame{
O Miguel tem uma mesada de €120 que pode usar para o consumo mensal de bolos e
maçãs. Assuma que um bolo (x) custa €2 e uma maçã (y) €1 e que as suas preferências
podem ser descritas pela função utilidade 𝑈 = √𝑥𝑦.
}

\frame{
a) Determine analiticamente a restrição orçamental do Miguel e faça a
representação gráfica do espaço das possibilidades de consumo.
}

\frame{
b) Represente no gráfico anterior o efeito de uma diminuição do preço dos bolos
para €1.5 na restrição orçamental. Em quanto aumentou a quantidade máxima
que o Miguel pode comprar de bolos? E de maçãs?
}

\frame{
c) Sabemos que o Miguel pode consumir um cabaz com 20 bolos e outro cabaz
com 30 bolos aos novos preços. Quantas maçãs está o Miguel a consumir em
cada um destes cabazes se ambos esgotarem o rendimento do Miguel? Será que
são indiferentes? Quantas maçãs teriam os cabazes se fossem indiferentes? Neste
caso ambos poderiam esgotar o orçamento?
}

\frame{
d) Calcule a taxa marginal de substituição entre os cabazes da alínea c) e interprete
o seu significado.
}

\frame{
e) Derive a taxa marginal de substituição (TMS) a partir da função utilidade
apresentada. Qual o valor da TMS no cabaz (𝑥, 𝑦) = (40,40)? Será que se trata
do cabaz de escolha óptima? Justifique.
}

\frame{
f) Recorrendo à 2ª Lei de Gossen, determine o cabaz de escolha óptima aos preços
iniciais.
}

\frame{
g) Dê um exemplo de um cabaz indiferente ao óptimo. Será que pode pertencer ao
espaço das possibilidades de consumo? Justifique.
}

\frame{
h) Determine a equação que descreve a curva de indiferença que contém o cabaz de
escolha óptima.
}

\frame{
i) Se o preço dos bolos aumentar para €2.5, o que espera que aconteça à
quantidade consumida deste bem?
}

\end{document}
