\documentclass[]{beamer}
\usepackage{amsmath}

\author[P. Fagandini]{Paulo Fagandini}
\institute[ISCAL-IPL]{Lisbon Accounting and Business School}

\begin{document}

\frame{
\maketitle
}

\frame{
1. Qual \'e o conceito de ``produto m\'edio'' na fun\c c\~ ao de produ\c c\~ ao?
\begin{enumerate}
    \item O produto total dividido pelo n\'umero total de unidades produzidas.
    \item O produto total dividido pelo n\'umero total de unidades de um \'unico fator de produ\c c\~ ao.
    \item O produto adicional obtido com a adição de uma unidade adicional de um fator variável.
    \item O custo médio da produção de uma unidade adicional.
\end{enumerate}
\pause Solu\c c\~ao: 2
}

\frame{
2. O que representa o ``produto marginal'' na função de produção?
\begin{enumerate}
    \item O produto total dividido pelo número total de unidades produzidas.
    \item O produto adicional obtido com a adição de uma unidade adicional de um fator variável.
    \item O produto total dividido pelo número total de unidades de um único fator de produção.
    \item O custo médio da produção de uma unidade adicional.
\end{enumerate}
\pause Solu\c c\~ao: 2
}

\frame{
3. Qual das seguintes afirmações é verdadeira sobre a distinção entre curto prazo e longo prazo na função de produção?
\begin{enumerate}
    \item No curto prazo, todos os fatores de produção são variáveis; no longo prazo, pelo menos um fator é fixo.
    \item No longo prazo, todos os fatores de produção são variáveis; no curto prazo, pelo menos um fator é fixo.
    \item Não há distinção entre curto e longo prazo na função de produção.
    \item No curto prazo, todos os fatores de produção são fixos; no longo prazo, todos os fatores são variáveis.
\end{enumerate}
\pause Solu\c c\~ao: 2
}

\frame{
4. Na função de produção, qual das seguintes etapas é caracterizada por um produto marginal decrescente?
\begin{enumerate}
    \item Etapa I.
    \item Etapa II.
    \item Etapa II e III.
    \item N\~ao h\'a rela\c c\~ao entre as etapas e o produto marginal.
\end{enumerate}
\pause Solu\c c\~ao: 3
}

\frame{
5. O que é o ``óptimo técnico'' na função de produção?
\begin{enumerate}
    \item O ponto onde a empresa maximiza seus lucros.
    \item O ponto onde a empresa minimiza seus custos totais.
    \item O nível de produção onde a empresa utiliza todos os seus recursos de forma mais eficiente possível.
    \item O ponto onde a empresa alcança seu produto médio máximo.
\end{enumerate}
\pause Solu\c c\~ao: 4
}

\frame{
6. Qual é a relação entre o custo marginal e o produto marginal do trabalho?
\begin{enumerate}
    \item Quando o custo marginal é maior que o produto marginal, a empresa está em uma situação de lucro.
    \item Quando o custo marginal é menor que o produto marginal, a empresa está em uma situação de lucro.
    \item O custo marginal é inversamente proporcional ao produto marginal no curto prazo.
    \item Não há relação entre o custo marginal e o produto marginal do trabalho.
\end{enumerate}
\pause Solu\c c\~ao: 3
}

\frame{
7. Como se relaciona o custo variável médio com o produto médio do trabalho?
\begin{enumerate}
    \item Quando o custo vari\'avel m\'edio \'e m\'inimo o produto m\'edio atinge o valor do produto m\'edio no $L$ do \'otimo t\'ecnico.
    \item Quando o custo variável médio é maior que o produto médio do trabalho, os custos estão diminuindo.
    \item Quando o custo variável médio é igual ao produto médio do trabalho, os custos estão minimizados.
    \item Não há relação entre o custo variável médio e o produto médio do trabalho.
\end{enumerate}
\pause Solu\c c\~ao: 1
}

\frame{
8. Na Etapa III da função de produção, o que acontece com o produto marginal?
\begin{enumerate}
    \item Aumenta constantemente en valor absoluto.
    \item Diminui constantemente.
    \item Aumenta em valor absoluto.
    \item Diminui em valor absoluto.
\end{enumerate}
\pause Solu\c c\~ao: 3
}

\frame{
9. Qual é o objetivo da empresa em relação ao custo total?
\begin{enumerate}
    \item Minimizar o custo total.
    \item Maximizar o lucro.
    \item Atingir um equilíbrio entre custo total e receita total.
    \item Não há objetivo específico em relação ao custo total.
\end{enumerate}
\pause Solu\c c\~ao: 2
}

\frame{
10. O que representa o custo marginal na função de custo?
\begin{enumerate}
    \item O custo adicional de produzir uma unidade adicional de produto.
    \item O custo total dividido pelo número total de unidades produzidas.
    \item O custo médio da produção de uma unidade adicional.
    \item O custo fixo de produção.
\end{enumerate}
\pause Solu\c c\~ao: 1
}

\frame{
11. Como é calculado o custo fixo médio?
\begin{enumerate}
    \item Dividindo o custo total fixo pelo número de unidades produzidas.
    \item Dividindo o custo variável total pelo número de unidades produzidas.
    \item Dividindo o custo total pelo número de unidades produzidas.
    \item Não é possível calcular o custo fixo médio.
\end{enumerate}
\pause Solu\c c\~ao: 1
}

\frame{
12. O que representa o custo variável médio?
\begin{enumerate}
    \item O custo total variável dividido pelo número de unidades produzidas.
    \item O custo total fixo dividido pelo número de unidades produzidas.
    \item O custo total variável dividido pelo custo fixo total.
    \item O custo total dividido pelo número de unidades produzidas.
\end{enumerate}
\pause Solu\c c\~ao: 1
}

\frame{
13. O que \'e a ``produ\c c\~ao unit\'aria'' na fun\c c\~ao de produ\c c\~ ao?
\begin{enumerate}
    \item A quantidade de produto produzida por cada unidade de fator variável.
    \item A quantidade de produto produzida quando todos os fatores de produção são variáveis.
    \item A quantidade de produto produzida quando todos os fatores de produção são fixos.
    \item A quantidade de produto produzida por cada unidade de fator fixo.
\end{enumerate}
\pause Solu\c c\~ao: 1
}

\frame{
14. Considere a seguinte função de produção no curto prazo: $Q = 10L - 0.5L^2$, onde $Q$ é a quantidade produzida e $L$ é a quantidade de trabalho utilizada. Qual é o produto marginal de trabalho quando $L = 5$ unidades?
\begin{enumerate}
    \item 10 unidades
    \item 5 unidades
    \item 15 unidades
    \item 25 unidades
\end{enumerate}
\pause Solu\c c\~ao: 2
}

\frame{
15. Na mesma função de produção do exemplo anterior ($Q = 10L - 0.5L^2$), qual é o produto médio de trabalho quando $L = 6$ unidades?
\begin{enumerate}
    \item 4 unidades
    \item 5 unidades
    \item 6 unidades
    \item 7 unidades
\end{enumerate}
\pause Solu\c c\~ao: 4
}

\frame{
16. Se o produto marginal de trabalho é 8 unidades e o produto médio de trabalho é 6 unidades, qual das seguintes afirmações é falsa?
\begin{enumerate}
    \item Se o sal\'ario for 16, o custo marginal \'e 2.
    \item Se o sal\'ario for 18, o custo vari\'avel m\'edio \'e 3.
    \item Se o sal\'ario for 16, o custo vari\'avel m\'edio \'e 2. % Escolha certa
    \item Certamente a empresa n\~ao se encontra no \'otimo t\'ecnico.
\end{enumerate}
\pause Solu\c c\~ao: 3
}

\frame{
    17. Na função de produção $Q = 10L - 0.5L^2$, qual é o nível de trabalho que corresponde ao ``óptimo técnico''?
    \begin{enumerate}
    \item $L = 5$
    \item $L = 10$
    \item $L = 15$
    \item $L = 0$
\end{enumerate}
\pause Solu\c c\~ao: 4
}

\frame{
18. Se a função de custo total é $C = 100 + 5Q + 0.1Q^2$, onde $Q$ é a quantidade produzida, qual é o custo marginal quando $Q = 10$ unidades?
\begin{enumerate}
    \item \$5
    \item \$5.5
    \item \$6
    \item \$7
\end{enumerate}
\pause Solu\c c\~ao: 4
}

\frame{
19. Se a função de custo total médio é $CM =\frac{10}{Q} + 20 + 0.1Q$, onde $Q$ é a quantidade produzida, qual é o custo variável médio quando $Q = 50$ unidades?
\begin{enumerate}
    \item \$20
    \item \$25
    \item \$30
    \item \$35
\end{enumerate}
\pause Solu\c c\~ao: 2
}

\frame{
20. Se o custo fixo é de \$200 e o custo variável total é de \$300, qual é o custo médio quando a quantidade produzida é de 100 unidades?
\begin{enumerate}
    \item \$2.00
    \item \$3.00
    \item \$4.00
    \item \$5.00
\end{enumerate}
\pause Solu\c c\~ao: 4
}

\end{document}