\begin{frame}
	\frametitle{Defesa de Concorr\^encia: O que \'e?}
	\begin{aquote}{Massimo Motta}
		Conjunto de pol\'iticas e leis que garantem que a concorr\^encia no mercado n\~ao \'e restringida de forma a que se reduza o bem-estar social.
	\end{aquote}
	
	\vspace{0.5cm}

	\begin{itemize}
		\item O bem-estar social \'e o objectivo a atingir com a pol\'itica de concorr\^encia.
		\item Tem particular relev\^ancia em mercados onde as empresas t\^em poder de mercado e onde a concorr\^encia \'e vi\'avel!
	\end{itemize}
\end{frame}

\begin{frame}
	\frametitle{Por que raz\~ao \'e necess\'aria a Defesa da Concorr\^encia?}
	Mesmo em mercados que funcionariam concorrencialmente, as for\c cas de mercado poderiam n\~ao levar ao resultado eficiente porque:
	\begin{itemize}
		\item As empresas podem comportar-se estrategicamente
		\item Podem criar ou fortalecer posi\c c\~oes dominantes atrav\'es de opera\c c\~oes de concentra\c c\~ao
		\item Podem efetuar ac\c c\~oes que aumentem os lucros e reduzam o bem-estar social: conluio, comportamento predat\'orio
	\end{itemize}
\end{frame}

\begin{frame}
	\frametitle{Conluio}
	\begin{itemize}
		\item Comportamento concertado de empresas (carteliza\c c\~ao, conluio) corresponde ao estabelecimento, por via de um acordo, de:
		\begin{itemize}
			\item Pre\c cos superiores a um padr\~ao;
			\item Quotas de mercado;
			\item Divis\~ao de mercados
		\end{itemize}
		\item Este acordo pode ser expl\'icito ou impl\'icito(conluio t\'acito)
		\item O acordo permite \`as empresas envolvidas usufruir de poder de mercado qeu de outra forma n\~ao teriam.
	\end{itemize}
\end{frame}

\begin{frame}
	\frametitle{Detec\c c\~ao de acordos entre empresas}
	\begin{itemize}
		\item A dissuas\~ao depende do n\'ivel das penas e o conluio \'e sujeito a pesadas penas: multas, pagamento de indemniza\c c\~oes e nos EUA at\'e penas de pris\~ao. Mas o valor esperado da pena \'e respetivo valor vezes a probabilidade de detec\c c\~ao...
		\item A melhor pol\'itica face ao conluio \'e criar mecanismos que tornem dif\'icil a emerg\^encia ou a sustentabilidade do acordo
	\end{itemize}
\end{frame}

\begin{frame}
	\frametitle{Abusos de Posi\c c\~ao Dominante}
	\begin{itemize}
		\item Um comportamento \'e predat\'orio se tem como objetivo proteger ou aumentar o poder de mercado de uma empresa dominante, atrav\'es da exclus\~ao ou elimina\c c\~ao de concorrentes por raz\~oes que n\~ao a sua efici\^encia.
		\item A exclus\~ao pode fazer-se atrav\'es da pr\'atica de pre\c cos baixos pela empresa dominante que baixem as receitas dos concorrentes.
	\end{itemize}
\end{frame}

\begin{frame}
	\frametitle{Pr\'aticas Predat\'orias}
	\begin{itemize}
		\item Direitos exclusivos de acesso a inputs
		\item Recusa de acesso a infra-estruturas essenciais
		\item Dumping
	\end{itemize}
\end{frame}

\begin{frame}
	\frametitle{Autoridade da Concorr\^encia-Lei da Concorr\^encia}
	\footnotesize{Lei 18/3004, Artigo 4\textsuperscript{\underline{o}}, Pr\'aticas proibidas:}
	
	\begin{enumerate}
		{\scriptsize
		\item S\~ao proibidos os acordos entre empresas, as decis\~oes de associa\c c\~oes de empresas e as pr\'aticas concertadas entre empresas, qualquer que seja a forma que revistam, que tenham por objeto ou como efeito impedir, falsear ou restringir de forma sens\'ivel a concorr\^encia no todo ou em parte do mercado nacional, nomeadamente os que se traduzam em:
		\begin{enumerate}[a)]
			{\scriptsize
			\item Fixar, de forma direta ou indireta, os pre\c cos de compra ou de venda ou interferir na sua determina\c c\~ao pelo livre jogo do mercado, induzindo, artificialmente, quer a sua alta quer a sua baixa;
			\item Fixar, de forma direta ou indireta, outras condi\c c\~oes de transac\c c\~ao efetuadas no mesmo ou em diferentes est\'adios do processo econ\'omico;
			\item Limitar ou controlar a produ\c c\~ao, a distribui\c c\~ao, o desenvolvimento t\'ecnico ou os investimentos;
			\item Repartir os mercados ou as fontes de abastecimento;
			\item Aplicar, de forma sistem\'atica ou ocasional, condi\c c\~oes discrminat\'orias de pre\c co ou outras relativamente a presta\c c\~oes equivalentes;
			\item Recusar, direta ou indiretamente, a compra ou venda de bens e a presta\c c\~oe de servi\c cos;
			\item Subordinar a celebra\c c\~ao de contratos \`a aceita\c c\~ao de obriga\c c\~oes suplementares que, pela sua natureza, ou segundo os usos comerciais, n\~ao tenham liga\c c\~ao com o objeto desses contratos.
			}
		\end{enumerate}
		}
	\end{enumerate}
\end{frame}